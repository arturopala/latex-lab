\subsection{Storia dei sistemi operativi GNU/Linux}

Non tutti sanno che, la storia dei sistemi operativi basati su kernel Linux comincia quasi 8 anni prima del rilascio del kernel stesso Nel 1983, \textbf{Richard Stallman}, un noto programmatore ed attivista presso il MIT di Cambridge (Massachusetts, USA), lancia il progetto GNU con l'intento di creare una sistema operativo, che prenderà il nome di sistema operativo GNU (acronimo ricorsivo di G.nu is N.ot U.nix), simile a Unix, ma privo delle sue limitazioni in fatto di licenze. Nel 1984 viene fondata la Free Software Foundation, con lo scopo di promuovere lo sviluppo e l'uso del software libero in tutto il mondo. Quest'ultima, nel 1989, scrive la prima versione (v1) della GNU General Public Licence (v2 1991, e v3 2007), una delle licenze di software libero più utilizzate al mondo.

Nonostante il lavoro processe bene, e si fosse arrivati ad ottenere dei buoni risultati sul piano delle applicazioni cosidette userspace, si registrava ancora un ritardo per quanto riguardava lo sviluppo del kernel, nucleo del sistema operativo. A colmare tale lacuna, nel 1991, arrivò l'allora ventitrenne studente di Helsinki, \textbf{Linus Torvalds}, che a titolo hobbystico pubblico il suo kernel (originariamente pensato per ambiente Minix ed iniziale nominato di FreeX). Di li a poco il kernel avrebbe preso il nome di Linux, e, rilasciato in licenza GNU GPLv2, si sarebbe svincolato dal sistema Minix. L'integrazione di questi due componenti, sistema operativo GNU e kernel Linux, ha dato vita negli anni allo sviluppo di distribuzioni GNU/Linux. 

Gia dal 1993, numerosi singoli sviluppatori ed aziende hanno cominciato a sviluppare le loro versioni di distribuzioni GNU/Linux; a fronte di numerosi prodotti hobbystici (es. Slackware) o interamente sviluppati da community (es. Debian), nascevano alcuni prodotti dedicati al mondo dell'azienda e dell'affidabilità, quali ad esempio \textbf{Red Hat} e \textbf{SUSE}.

Ad oggi non è calcolabile con precisione il numero di distribuzioni GNU/Linux attive nel panorama FOSS (Free and Open Source Software) ma di fatto, buona parte di questo panorama è di derivazione diretta o indiretta da queste prime progenitrici.

Di fatto, quello che una volta si credeva essere un sistema operativo alle prime armi e scarsamente utilizzabile, ad oggi ha le massime percentuali di utilizzo in tutti qui contesti dove l'affidabilità, le performance e la sicurezza sono fondamentali, come ad esempio in numerosissimi contesti enterprise. 

